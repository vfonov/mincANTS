ANTS supports both volumetric registration and point set registration. The image / point set similarity metrics in ANTS are unified in the form of a function on the images or the point sets:
$$\textbf{Simlarity}[\textbf{fixedImage}, \textbf{movingImage}, \textbf{weight}, \textbf{parameters}]$$

The similarity type for the deformation transformation is specified by -m option. This option contains two parts: simarity type and the parameters inside the brackets. (Note: no white spaces exist between parameters.) The possible similarity metrics are: 
\begin{itemize}
 \item Intensity based:
    \begin{itemize}
        \item Cross correlation

        -m CC/cross-correlation/CrossCorrelation[fixedImage,movingImage,weight,radius]        
        
        This metric works well for intra-modality image registration. 
        
        Example: \textbf{-m CC[fixed.nii,moving.nii,1,5]} specifies:
            \begin{itemize}
             \item the fixed image: fixed.nii
             \item the moving image: moving.nii             
             \item weight for this metric is 1 (i.e. only this metric drives the registration).      \item the region radius for computing cross correlation is 5
            \end{itemize}

        \item Mutual information

        -m MI/mutual-information/MutualInformation[fixedImage,movingImage,weight,number of histogram bins]
        
        This metric works both well for intra-modality and inter-modality image registration. 
        
        Example: \textbf{-m MI[fixed.nii,moving.nii,1,32]} specifies:
            \begin{itemize}
             \item the fixed image: fixed.nii
             \item the moving image: moving.nii             
             \item weight for this metric is 1 (i.e. only this metric drives the registration).      \item the number of bins in computing mutual information is 32
            \end{itemize}
                     
        \item PR

        -m PR/probabilistic/Probabilistic[fixedImage,movingImage,weight,radius]
        
        TODO: This metric works for intra-modality image registration.
        
        Example: \textbf{-m PR[fixed.nii,moving.nii,1,5]} specifies:
            \begin{itemize}
             \item the fixed image: fixed.nii
             \item the moving image: moving.nii             
             \item weight for this metric is 1 (i.e. only this metric drives the registration).      \item the region radius for computing cross correlation is 5
            \end{itemize}

        \item Mean square difference

        -m MSQ/mean-squares/MeanSquares[fixedImage,movingImage,weight, 0]
        
        This metric works for intra-modality image registration.
        
        Example: \textbf{-m MSQ[fixed.nii,moving.nii,1,0]} specifies:
            \begin{itemize}
             \item the fixed image: fixed.nii
             \item the moving image: moving.nii             
             \item weight for this metric is 1 (i.e. only this metric drives the registration).      \item 0 is a padding value of no real meaning.
            \end{itemize}

    \end{itemize}

 \item Point set based:
    \begin{itemize}
        \item Point set expectation

        -m PSE/point-set-expectation/PointSetExpectation [fixedImage,movingImage,fixedPoints,movingPoints,weight,pointSetPercentage,pointSetSigma, boundaryPointsOnly,kNeighborhood,PartialMatchingIterations=100000]

        the partial matching option assumes the complete labeling is in the first set of label parameters ... more iterations leads to more symmetry in the matching  - 0 iterations means full asymmetry 

        TODO: Example: \textbf{-m JTB[fixed.nii,moving.nii,fixed.nii,moving.nii,?????]} specifies:
            \begin{itemize}
             \item the fixed image: fixed.nii
             \item the moving image: moving.nii             
             \item the fixed point set: fixed.nii
             \item the moving point set: moving.nii
             \item weight for this metric is 1 (i.e. only this metric drives the registration).
             \item TODO: What are the parameters?
            \end{itemize}


        \item Jensen-Tsallis BSpline

        -m JTB/jensen-tsallis-bspline/JensenTsallisBSpline [fixedImage,movingImage,fixedPoints,movingPoints,weight,pointSetPercentage,pointSetSigma, boundaryPointsOnly,kNeighborhood,alpha,meshResolution,splineOrder,numberOfLevels, useAnisotropicCovariances]

        the partial matching option assumes the complete labeling is in the first set of label parameters ... more iterations leads to more symmetry in the matching  - 0 iterations means full asymmetry 

            TODO: Example: \textbf{-m JTB[fixed.nii,moving.nii,fixed.nii,moving.nii,?????]} specifies:
            \begin{itemize}
             \item the fixed image: fixed.nii
             \item the moving image: moving.nii             
             \item the fixed point set: fixed.nii
             \item the moving point set: moving.nii
             \item weight for this metric is 1 (i.e. only this metric drives the registration).
             \item TODO: What are the parameters?
            \end{itemize}

    \end{itemize}


 \item For the affine registration, there are two types of similarity metrics, which are specified using \textbf{$--$affine-metric-type}:
    
    \begin{itemize}
     
    \item Mutual information, specified using \textbf{MI}.
         Example: \textbf{$--$affine-metric-type MI}. Options can be specified as:         
           
         \begin{itemize}
            \item \textbf{$--$MI-option} specifies the number of bins and samples: MI\_bins x MI\_samples, for example: \textbf{$--$MI-option 32x8000}
         \end{itemize}
    
    \item Mean square difference, specified using \textbf{MSE}.
         Example: \textbf{$--$affine-metric-type MSE}
         \begin{itemize}
            \item \textbf{$--$MI-option} specifies the number of bins and samples: MI\_bins x MI\_samples, for example: \textbf{$--$MI-option 32x8000}
        \end{itemize}


    \end{itemize}

\end{itemize}

