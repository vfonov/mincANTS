% pdflatex !!!  pdf2ps !!! a2ps

%\documentclass[12pt]{article}
\documentclass[preprint,authoryear,12pt]{elsarticle}
%\documentclass[10pt,twocolumn]{article}
\usepackage{times,verbatim}
\pagestyle{empty}

% Set dimensions of columns, gap between columns, and paragraph indent 
\setlength{\textheight}{9in}
\setlength{\textwidth}{7in}
\setlength{\topmargin}{0in}
\setlength{\headheight}{0in}
\setlength{\headsep}{0in}
\setlength{\parindent}{1pc}
\setlength{\oddsidemargin}{-.1875in}  % Centers text.
\setlength{\evensidemargin}{-.1875in}

\newcommand{\FIGUREDIR}{/data/Avants/}
\newcommand{\FTDNUM}{25~}
\newcommand{\ADNUM}{24~}
\newcommand{\ELDNUM}{23~}
\newcommand{\x}{{\bf x}}
\newcommand{\y}{{\bf y}}
\newcommand{\I}{{\bar I}}
\newcommand {\alg}{{\em Atropos}}
\usepackage{amsmath, amsthm, amssymb}
\usepackage{epsfig}
\usepackage{graphics}
\usepackage{soul}
\usepackage{rotating}
\usepackage{multirow,booktabs,ctable,array}
\usepackage{listing}						
\usepackage{listings}					
\DeclareMathOperator*{\argmin}{arg\,min}
\DeclareMathOperator*{\argmax}{arg\,max}

\begin{document}

\date{}
\begin{frontmatter}
%\title{Evaluation of a Prior-Constrained Registration Initialized Template Assisted Segmentation Algorithm for $N$-Classes}
\title{NIREP and LPBA Evaluation of an Open-Source Framework for Cortical Parcellation}
\author{Nick Tustison, Brian Avants\footnote{First two authors equal co-authorship}, Philip A. Cook and James C. Gee\\  
  Penn Image Computing and Science Laboratory 
	\\ University of Pennsylvania \\ Philadelphia, PA~~19104-6389
	}

\begin{abstract} 
Neuroanatomical coordinate systems are essential for the
interpretation of structural and functional imaging studies.  This
work proposes a fully image-based, open-source approach to label-based
cortical parcellation that uses existing training data to identify the
same structures in new datasets.  The driving engine underlying this
system is a new N-class EM-MRF algorithm, \alg (distributed with the
Advanced Normalization Tools (ANTs) software), that incorporates
spatial priors to guide the segmentation.  \alg efficiently implements
a variety of regularization and statistical models for segmentation
and uses an [ EM or ICA ] energy minimization approach.  \alg's
efficient implementation allows one to solve {\em many class}
segmentation problems (here up to 56 classes are used) while using
both spatial probability maps and either geodesic or Euclidean
distance priors to constrain the segmentation.  This work evaluates
\alg performance on the parcellation problem given two different
training datasets, the LPBA40 dataset from UCLA and the NIREP dataset
from University of Iowa.  We evaluate three aspects of the problem: (1) the quality
of automatic cortical extraction from raw data; (2) the quality of
parcellation, given a ground truth cortical extraction; (3) the
quality of automated cortical extraction followed by automated
cortical parcellation.  Component (3) corresponds to the most
realistic clinical case, while component (2) eliminates the confound
of defining the cortex itself and focuses only on the parcellation
problem.  We also quantify the effect of using either a parametric or
non-parametric appearance model in the segmentation.  Finally, we
distribute all of the scripts and code from this evaluation.
\end{abstract}
\begin{keyword}
segmentation \sep expectation maximization \sep spatial prior \sep brain 
\end{keyword}
\end{frontmatter}

\section{Introduction}
Review segmentation methods , in particular Warfield 2009, FAST and
other methods that use spatial priors.  Review evaluation of
segmentation.

The expectation maximization and markov random field framework
(EM-MRF) for segmentation is reliable and efficient for large-scale
neuroimage processing in a variety of conditions.  However, EM-MRF
methods are sensitive to initial conditions which may lead to
performance instability across clinical subjects and in longitudinal
studies.  This paper proposes an open-source $N$-class extension to
the EM-MRF model, \alg, that includes strong use of template-based
information in both the initialization and as spatial priors in order
to guide the method into a consistent local minimum.

\begin{comment}{
  We evaluate this
method on BrainWeb 20 data and show that by relying upon accurate,
high-resolution diffeomorphic image registration, \alg~is capable of
reliably segmenting the BW20 whole head data into separate tissue
classes for muscle, skin, skull and bone marrow as well as the
standard three tissues, cerebrospinal fluid and gray and white matter.
\alg's three-tissue performance compares favorably with the standard
EM-MRF algorithm, FAST.  We also use the OASIS dataset to show that
\alg~increases segmentation repeatability relative to FAST.  The
evaluation data, the data-derived templates, the code and application
itself are publicly available.
}
\end{comment}

One advantage of the ANTS-Atropos pipeline is that we parcellate
cortex entirely in the image space, thus avoiding the difficulty of
transferring labels from the mesh space back to the image space---a
problem that is a confound of surface-based methods\cite{Klein2010}.


\section{Methods} \alg~encodes a family of segmentation techniques
that may be instantiated for different applications.  We describe the
general theory, algorithm and implementation that form \alg~and then
specify the particular parameters used for the applications of
interest to this research.

\subsection{Notation} An image, $I$, maps a domain, $\Omega$ into the
positive real numbers, such that $I \colon \Omega \rightarrow
\mathbb{R}^+$.  The goal of segmentation, in general, is to define the
spatial distribution of a finite set of labels over this domain.  We
denote the segmentation itself as $\eta \colon \Omega \rightarrow L$
where $L = \{ L_1 = 1 , \cdots , L_N=N \}$, a set of integer indexed
segmentation labels.  Note that $\eta$ may be formed as $\eta =
\sum_{i=1}^{i=N} L_i \eta_i $ where $\eta_i$ is the binary
segmentation for label $L_i$.  A prior estimate for the label image
$\eta_i$ is here denoted $\eta^s_i$ with a complete set of priors
denoted $\eta^s$.  The boundary of the binary segmentation -- where a
$0/1$ transition edge exists -- is denoted $\partial \eta^s_i$, for
the prior, and $\partial \eta_i$ for the label image.

\subsection{Apocrita Theory} A general maximum a posteriori criterion
for segmentation seeks,
\begin{equation} {\hat \eta} = \argmax _{\eta} \Pr( \eta | I )(\x) =
\Pr( I | \eta )(\x) \Pr(\eta)(\x),
\end{equation} where $I$ is the input image, ${\eta}$ represents the
label set configuration taken from the set $L$, $\Pr$ is the
probability, $\x$ is the spatial index and the optimal solution is
$\hat \eta$.  The input image $I$, here, is an unlabeled T1 MRI
indexed by the value $\x \in \Omega$ where $\Omega$ is the image's
spatial domain.  This probability is composed of the likelihood (first
term) and the prior (second term).  \alg~ uses a spatially varying
likelihood term and a two-component prior term that takes into account
both spatial distribution and label smoothness, the latter via a
standard MRF prior.

The likelihood term for a single label value $L_i \in L$ is,
\begin{eqnarray} \Pr( I | \eta_i )(\x)=\frac{1}{Z_i} \exp(- \| I(\x) -
\mu_i(\x) \|^2 / \sigma_i^2 ),
\end{eqnarray} where $Z_i$ is a normalizing constant, $\mu(\x)$ is a
spatially varying estimate of the tissue mean and $\sigma$ is a
standard deviation.  The prior term is given by,
\begin{eqnarray} \Pr(\eta_i)(\x) = \frac{1}{Z^\prime_i} \exp( -f( \x -
\y_{\partial \eta^s_i} ) / \sigma_{\eta^s_i}^2 ) p(\eta_i |
\eta^\aleph_i), \\ \notag f( \x - \y_{\partial \eta^s_i} ) = (1 -
\eta^s_i (\x) ) \| \x - \y_{\partial \eta^s_i} \|,
\end{eqnarray} where $p(\eta_i | \eta^N_i)$ is the MRF smoothness
probability based on the local neighborhood $\aleph$, the $Z$ is a
normalizing constant and $\y_{\partial \eta^s_i}$ is the nearest point
to $\x$ on the boundary of this labeling. \alg~requires a user or
template-defined $\eta^s_i$ and the standard deviation
$\sigma_{\eta^s_i}$ if a non-unity spatial prior component is desired
for that label.  The free parameters, that must be estimated
iteratively, are therefore $\mu_i, \sigma_i$ and the label set itself
$\hat \eta$, which defines the (locally) optimal spatial distribution
of $L$ through $\Omega$.  Figure~\ref{fig:spatp} shows the
distribution of the spatial prior as a function of the distance from
prior-defined object boundary.  Note that $f$ may easily be varied for
other applications or that fixed probability images may also be
substituted here.  This choice of $f$ is motivated by the fact that it
allows compressed storage of the priors in a single image, $\eta^s$,
while also maintaining the ability to manipulate -- for each
$\eta^s_i$ -- the spatial influence of the prior via
$\sigma_{\eta^s_i}$.  Practically, this is especially valuable when,
$N$, the number of labels, is large.

\subsection{Apocrita EM Algorithm}
The EM approach. 

\subsection{Apocrita Implementation}
The command line and practical example, ITK, etc. 
Template-based initialization.  



\subsection{The LPBA40 Dataset}  
\label{sec:lpba}
The LPBA40 dataset \cite{Shattuck2008} was collected at the North Shore Long
Island Jewish Health System imaging center and is maintained at UCLA.
LPBA40 contains 40 images (20 male $+$ 20 female) from normal, healthy
ethnically diverse volunteers with average age of 29.2 $\pm$ 6.3
years.  Each subject underwent 3D SPGR MRI on a 1.5T GE system
resulting in $0.86 \times 0.86 \times 1.5 mm^3$ images.  Each MRI in
the LPBA40 dataset was manually labeled with 56 independent structures
at the UCLA Laboratory of Neuro Imaging (LONI).  The test-retest
reliability of the labeling, across raters, was reported as a minimum
Jaccard ratio of $0.697$ in the supramarginal gyrus to a maximum of
$0.966$ in the gyrus rectus.  A single labeling of each image is made
available to the public and used, here, as silver-standard data 
for both training and testing in our cross-validation scheme. 

\subsection{The NIREP Dataset}  
The non-rigid image registration evaluation project (NIREP
http://www.nirep.org/) is a resource of 16 high quality labeled brain
images at 1$mm^3$.  Each brain was labeled with 32 cortical regions
(16 on each hemisphere) and an additional class of other gray matter
tissue.  Regions vary in size from large (inferior temporal region) to
small (temporal pole, insula gyrus, frontal pole).  The main drawback
is a lack of inter-rater reliability numbers -- in particular because
visual inspection and comparison of labelings reveals a degree of
inconsistency in labeling of particular regions across subjects.
Nevertheless, the NIREP dataset is perhaps the highest quality
evaluation dataset currently available for the cortex. 

\begin{figure}
%\center{ \includegraphics[width=3in]{figures/spatialprior.pdf} }
\caption{The spatial prior.
\label{fig:spatp} }
\vspace{-0.1in}
\end{figure}


\section{Results}
% \subsection{BWeb 20 Results} $N$-class segmentation and compare to FAST. 

% \subsection{Oasis Results} Review stability across repeat images vs. FAST.  

\section{Discussion}

\section{Conclusion}
To our knowledge, this is the first demonstration of a consistent and

\paragraph{Acknowledgments}
{This work was supported in part by NIH ... }

\bibliographystyle{elsarticle-harv.bst}
\bibliography{avantsAll,references} 

\end{document}

